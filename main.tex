\documentclass[12pt]{article}

\usepackage[text={5.5in,9in},centering,top=0.75in]{geometry}

\usepackage{xfrac}
\usepackage{libertine}
\usepackage{libertinust1math}
\usepackage[T1]{fontenc}

\usepackage{amsmath,amssymb,amsthm}

\title{Linear Algebra Notes}

\author{William Kephart}

\date{June 2022}

\pagestyle{empty}

\begin{document}
\maketitle
\begin{center}
    Chapter 1
\end{center}
\noindent\textbf{Definition:} A \textbf{linear equation} in the variables $x_1,...,x_n$ is an equation that can be written
in the form \begin{center}
$a_1x_1 + a_2x_2 + ... + a_nx_n = b$
\end{center}
where $b$ and the coefficients $a_1,...,a_n$ are real or complex numbers and $n$ may be any positive integer.\vspace{\baselineskip}

\noindent\textbf{Definition:} A \textbf{system of linear equations} (or a \textbf{linear system}) is a collection of one or more linear equations involving the same variables.\vspace{\baselineskip}

\noindent\textbf{Definition:} A \textbf{solution} of the system is a list ($s_1,s_2,...,s_n$) of numbers that makes each equation a true statement when substituted for $x_1,...,x_n$ respectively.\vspace{\baselineskip}

\noindent\textbf{Definition:} The set of all possible solutions is called the \textbf{solution set} of the linear system. Two linear systems are called \textbf{equivalent} if they have the same solution set.\vspace{\baselineskip}

\textit{Remark:} Finding the solution set of a system of two linear equations in two variables amounts to finding the intersection of two lines.\vspace{\baselineskip}

\noindent\textit{Fact:} A system of linear equations has either \begin{enumerate}
    \item no solution
    \item exactly one solution
    \item infinitely many solutions
\end{enumerate}
\textbf{Definition:}A system of linear equations is said to be \textbf{consistent} if it has either one solution or infinitely many solutions. A system is \textbf{inconsistent} if it has no solution.\vspace{\baselineskip}
\end{document}
